%% start of file `template.tex'.
%% Copyright 2006-2010 Xavier Danaux (xdanaux@gmail.com).
%
% This work may be distributed and/or modified under the
% conditions of the LaTeX Project Public License version 1.3c,
% available at http://www.latex-project.org/lppl/.

% Version: 20110122-4


\documentclass[11pt,a4paper,nolmodern]{moderncv}
\usepackage[onehalfspacing]{setspace}
\usepackage{bastelfreak}
\usepackage{fontspec}
\usepackage[english]{babel}
\linespread{1.18}
% for some reason, lines take up a lot of space in itemize in English...
\newenvironment{tightitemize}
   {\begin{itemize}
   \setlength{\parskip}{0pt}}
   {\end{itemize}}


% personal data
\title{DevOps Engineer}
\extrainfo{%
\xing~\httplink{xing.com/profile/Tim\_Meusel}\\%
\faGithub~\httplink{github.com/bastelfreak}\\%
\faCar~Driving License} % optional, remove the line if not wanted

\myquote{DevOps is all about empathy}{Rob Nelson}


%\nopagenumbers{}                             % uncomment to suppress automatic page numbering for CVs longer than one page
\begin{document}
\setmainfont{TeX Gyre Pagella}
%\setsansfont{Myriad Pro}

\hyphenpenalty=10000
\maketitle

\section{Skills}

\subsection{Expert Skills}
\cvcomputer{Configuration Management}{Puppet\maintainer, installimage\maintainer}
           {Operating Systems}{Archlinux\contributor, Gentoo}

\subsection{Development}
\cvcomputer{Languages}{C\#, Python, Shell/Bash, Ruby}
           {Frameworks}{RSpec, Beaker, Sinatra, Rails}
\cvcomputer{Databases}{MySQL, PostgreSQL}
           {Methods}{Object Oriented Programming, MVC, Unit Tests, UML}
\cvcomputer{Source Management}{Git, SVN}
           {Tools}{Bitbucket, GitHub, Travis CI, Jenkins}

\subsection{Systems and Networks Administration}
\cvcomputer{Web}{Apache, Lighttpd, Nginx, Varnish, Squid, proftpd, HAProxy}
           {Monitoring}{ELK-Stack, Zabbix, Munin, Prometheus, Graphite, Grafana, Graylog}
\cvcomputer{Mail}{Postfix, Dovecot, Amavis}
           {Backup}{Duply, Borg}
\cvcomputer{Installation / Deployment}{Kickstart, Puppet\maintainer, installimage\maintainer}
           {Infrastructure Software}{ISC-DHCPd, OpenVPN, hpa-tftpd, NFS, Marmoset, OpenLDAP}
\cvcomputer{Operating Systems}{GNU/Linux (Debian, Ubuntu, RedHat, Archlinux\contributor, Gentoo),
                               FreeBSD, pfSense}
           {Databases}{MySQL, PostgreSQL}
\cvcomputer{Virtualization}{VMWare, Docker, QEMU/KVM, Parallels Cloud Server}
           {Security}{SSL, PGP/GnuPG, OpenSSH}
\cvcomputer{DNS}{Unbound, Bind, Dnsmasq, PowerDNS}
           {Networking}{Cisco routing and switching, VLANs, Network design, IPv6, Network security}
\devnotes{Maintainer}{Contributor}

\newpage

\section{Experience}

\tlcventry{2014}{0}{DevOps Engineer}{\href{https://www.heg.com/}{Hosteurope GmbH} then GoGaddy EMEA}{Cologne}{}%
  {DevOps Engineer, System Administrator
\begin{itemize}
  \item DevOps Engineer:
    \begin{itemize}
      \item Operated a fleet of 2600+ Puppet-operated servers;
      \item Architectured and deployed a scaleable ELK-Stack with Graylog;
      \item Planned and executed migrations from Puppet version 3 to 4;
      \item Wrote and contributed to 100+ Puppet modules (mostly public), including plugins (facts, functions, types \& providers, indirectors), with unit- and acceptance tests;
      \item Designed and built a technical fabric to provision physical servers for multiple brands;
      \item Packaged software for Archlinux. Operated and improved the package build systems and package repositories.
    \end{itemize}
  \item System Administrator
    \begin{itemize}
      \item Operate multiple Cloud platforms with more than 3300 nodes. Provide third level support;
      \item Test and evaluate new hardware;
      \item Enhance the hardwaremonitoring to predict disk failures;
      \item Automate and optimize the consolidation of legacy platforms.
    \end{itemize}
\end{itemize}}

\tlcventry{2012}{2014}{Systems Engineer}{\href{https://hetzner.de/}{Hetzner Online AG}}{Falkenstein}{}{Leading a cloud platform, maintainting datacenter infrastructure
\begin{tightitemize}
 \item Rebuild of a Puppet infrastructure for 1500+ servers;
 \item Migrated servers from Puppet 2 to 3;
 \item Implemented cgroups for resource management in a cloud environment;
 \item Operated the base infrastructure of a datacenter with 200,000+ servers, including authoritative and recursive nameservers, installation environment, package mirrors, DHCP and NFS clusters;
 \item Maintained, developed and improved the current cloud platform with 20,000+ virtual instances;
 \item Planed and evaluated a new cloud platform with distributed and local storage;
 \item Provided third level support for the cloud environment and the datacenter infrastructure systems;
 \item Monitoring of the fleet of servers with zabbix;
 \item Writing of technical documentation.
\end{tightitemize}}

\tlcventry{2016}{0}{Vox Pupuli Project Management Committee}{\href{https://voxpupuli.org/}{Vox Pupuli}}{Internet}{}{Strategic Development, Leadership, Community Representative
\begin{tightitemize}
  \item Onboarding for new community members;
  \item Mitigated security issues in Puppet modules;
  \item Kept track of outdated software dependencies;
  \item Represented the community at events and conferences;
  \item Developed and realized a migration plan from Puppet 3 to 4;
  \item Escalated bugs and featurerequests from the community to Puppet Inc.
\end{tightitemize}}

\tlcventry{2015}{0}{Puppet Contributor}{\href{https://puppet.com/}{Puppet Inc}}{Internet}{}{Development, bugfix and documentation
\begin{tightitemize}
 \item Review and merge incoming patches;
 \item Deal with (Security-) Issues, bug reports, and feature requests;
 \item Migrate modules to puppet 4
\end{tightitemize}}

\tlcventry{2015}{0}{Developer and Maintainer}{\href{https://github.com/virtapi/virtapi}{VirtAPI-Stack}}{Internet}{}{Architecture design and development
\begin{tightitemize}
  \item Design and document an API for a cloud infrastructure;
  \item Maintain a public fork of the installimage;
  \item Create a live linux image for rescue operations(\href{https://github.com/virtapi/LARS}{LARS}).
\end{tightitemize}}

\section{Education}

\tllabelcventry{2014}{2018}{\kern-2em 2014--2018}{Staatlich geprüfter Informatiker}{\href{http://www.hhek.bonn.de/front_content.php?idcat=120}{Heinrich-Hertz-Europakolleg der Bundesstadt Bonn}}{}{}{}

\tllabelcventry{2009}{2012}{2009--2012}{Ausbildung zum Fachinformatiker Systemintegration}{}{Polizei NRW}{}{}{}

\tllabelcventry{2003}{2009}{2003--2009}{Fachoberschulreife mit Qualifikation}{}{Realschule Heessen}{}{}{}

\section{Certifications}

\tllabelcventry{2016}{2017}{\kern-3em 2016--2017}{Cisco Certified Network Associate, Part 1 and 2}{\href{https://www.cisco.com/c/en/us/training-events/training-certifications/certifications/associate/ccna-routing-switching.html}{Cisco Inc}}{}{}{}

\tldatecventry{2017}{ITIL{\textregistered} v3 Foundation Examination}{}{}{}{}{}

\tldatecventry{2013}{Puppet Fundamentals Administrator}{\href{https://puppet.com/support-services/training}{Puppet Inc}}{}{}{Puppet IT automation software}

\section{Publications}

\tldatecventry{2017}{Datawarehousing for Cloud Computing Metrics}{\href{https://bastelfreak.de/thesis-de.pdf}{bastelfreak.de/thesis-de.pdf}}{}{}{Analyse software to measure cloud metrics in a datawarehouse style}{}

\tllabelcventry{2015}{2017}{\kern-2em 2015--2017}{Network Design Principles}{\href{https://github.com/bastelfreak/talks}{github.com/bastelfreak/talks}}{}{}{Taught multiple lessons about network design in datacenters and carriers, including routing, switching, network-security and IPv6}

\tldatecventry{2015}{VirtAPI Paper}{\href{https://bastelfreak.de/virtapi.pdf}{bastelfreak.de/virtapi.pdf}}{}{}{Documentation about a self-designed API to manage a QEMU cloud environment with Puppet}{}

\tldatecventry{2012}{Einrichtung einer Testumgebung für Betriebssysteme und Softwareupdates}{\href{https://bastelfreak.de/Projektdokumentation.pdf}{bastelfreak.de/Projektdokumentation.pdf}}{}{}{Something with Gentoo}{}

\section{Foreign Languages}

\cvlanguage{English}{Fluent}{Daily practice, conferences given in English}

\end{document}
