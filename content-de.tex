%% start of file `template.tex'.
%% Copyright 2006-2010 Xavier Danaux (xdanaux@gmail.com).
%
% This work may be distributed and/or modified under the
% conditions of the LaTeX Project Public License version 1.3c,
% available at http://www.latex-project.org/lppl/.

% Version: 20110122-4


\documentclass[11pt,a4paper,nolmodern]{moderncv}
\usepackage[onehalfspacing]{setspace}
\usepackage{bastelfreak}
\usepackage{fontspec}
\usepackage[english]{babel}
\linespread{1.18}
% for some reason, lines take up a lot of space in itemize in English...
\newenvironment{tightitemize}
   {\begin{itemize}
   \setlength{\parskip}{0pt}}
   {\end{itemize}}


% personal data
\title{DevOps Engineer}
\extrainfo{%
\xing~\httplink{xing.com/profile/Tim\_Meusel}\\%
\faGithub~\httplink{github.com/bastelfreak}\\%
\faCar~Driving Licence} % optional, remove the line if not wanted

\myquote{DevOps is all about empathy}{Rob Nelson}


%\nopagenumbers{}                             % uncomment to suppress automatic page numbering for CVs longer than one page
\begin{document}
\setmainfont{TeX Gyre Pagella}
%\setsansfont{Myriad Pro}

\hyphenpenalty=10000
\maketitle



\section{Fähigkeiten}

\subsection{Kernfähigkeiten}
\cvcomputer{Entwicklung}{Puppet\maintainer, installimage\maintainer}
           {Operativer Bereich}{Archlinux\contributor, Gentoo, QEMU/KVM}

\subsection{Entwicklung}
\cvcomputer{Sprachen}{C\#, Python, Shell/Bash, Ruby}
           {Frameworks}{RSpec, Beaker, Sinatra, Rails}
\cvcomputer{Datenbanken}{MySQL, PostgreSQL}
           {Methoden}{Objektorientierte Programmierung, MVC, Unit Tests, UML}
\cvcomputer{Source Management}{Git, SVN}
           {Werkzeuge}{Bitbucket, GitHub, Travis CI, Jenkins}

\subsection{System- und Netzwerk-Administration}
\cvcomputer{Web}{Apache, Lighttpd, Nginx, Varnish, Squid, proftpd, HAProxy}
           {Monitoring}{ELK-Stack, Zabbix, Munin, Prometheus, Graphite, Grafana, Graylog, Collectd}
\cvcomputer{Mail}{Postfix, Dovecot, Amavis}
           {Backup}{Duply, Borg}
\cvcomputer{Installation / Deployment}{Bootstrapping, Kickstart, Puppet\maintainer, installimage\maintainer}
           {Infrastruktur Software}{ISC-DHCPd, OpenVPN, hpa-tftpd, NFS, Marmoset, OpenLDAP}
\cvcomputer{Betriebssysteme}{GNU/Linux (Debian, Ubuntu, RedHat, Archlinux\contributor, Gentoo),
                               FreeBSD, pfSense}
           {Datenbanken}{MySQL, PostgreSQL}
\cvcomputer{Virtualisierung}{VMWare, Docker, QEMU/KVM, libvirt, Parallels Cloud Server}
           {Security}{SSL, PGP/GnuPG, OpenSSH, SELinux, Threat Modeling, Server hardening}
\cvcomputer{DNS}{Unbound, Bind, Dnsmasq, PowerDNS}
           {Networking}{Cisco routing and switching, VLANs, Network design, IPv6, Network security}
\devnotes{Maintainer}{Contributor}

\newpage

\section{Erfahrung}

\tlcventry{2014}{0}{DevOps Engineer}{\href{https://www.heg.com/}{Hosteurope GmbH} später GoGaddy EMEA}{Köln}{}%
  {DevOps Engineer, System Administrator
\begin{itemize}
  \item DevOps Engineer:
    \begin{itemize}
      \item Betrieb von über 2600, mit Puppet administrierten, Servern;
      \item Architektur für einen ELK-Stack mit Graylog entwickelt und umgesetzt;
      \item Migration der gesammten Umgebung von Puppet 3 -> Puppet 4 -> Puppet 5  geplant und durchgeführt;
      \item Entwicklung und Erweiterung von über 100 Puppet Modulen (fast alle FOSS), inklusive Plugins, Unit- und Acceptance-Tests;
      \item Design und Entwicklung einer Fabric zur Firmen-unabhängigen Provisionierung physischer Server;
      \item Erstellung von Paketen diverser Software für Archlinux, Betrieb und Entwicklung eines automatischen Buildsystems mit eigenen Repositories;
      \item Product Owner für mehrere Infrastruktur Dienste.
    \end{itemize}
  \item System Administrator:
    \begin{itemize}
      \item Betrieb mehrerer Cloud-Platformen mit mehr als 3300 Hostsystemen, Third-Level-Support;
      \item Test und Evaluierung neuer Hardware bezüglich des Stromverbrauchs, Effizienz und der Packungsdichte der VMs;
      \item Optimierung des Hardware Monitorings um Festplatten Ausfällen vorzubeugen;
      \item Automatisierung und Optimierung der Konsolidierung alter Platformen;
      \item Entwicklung eines skalierbaren Monitoring Systems mit Autodiscovery auf Basis von Zabbix.
    \end{itemize}
\end{itemize}}

\tlcventry{2012}{2014}{Systems Engineer}{\href{https://hetzner.de/}{Hetzner Online AG}}{Falkenstein}{}{Verantwortlich für eine Cloud-Platform, Verwaltung von Services zum Betrieb eines Rechenzentrums
\begin{tightitemize}
  \item Neuentwicklung einer Puppet Infrastruktur für über 1500 Server;
 \item Migration von Puppet 2.6 und 2.7 auf Puppet 3.2, kontinuierliche Updates auf Puppet 3.7;
 \item Implementierung von Cgroups zur Ressourcenverwaltung in einer Cloud Umgebung;
 \item Betrieb der Basisinfrastruktur eines Rechenzentrums für über 200.000 Server, unter anderem autoritative und rekursive DNS-Cluster, PXE-Installationsumgebung, Paketmirror, DHCP- und NFS-Cluster;
 \item Verwaltung, Entwicklung und Optimierung der vorhandenen Cloud-Platform mit über 20.000 virtuellen Instanzen;
 \item Planung und Evaluierung einer neuen Cloud-Platform mit verteiltem und lokalem Storage;
 \item Third-Level-Support Für die Cloud-Platform sowie Rechenzentrums-Infrastruktur;
 \item Monitoring aller Services mit Zabbix;
 \item Erstellung technischer Dokumentation.
\end{tightitemize}}

\tlcventry{2016}{0}{Vox Pupuli Project Management Committee}{\href{https://voxpupuli.org/}{Vox Pupuli}}{Internet}{}{Community Management, Repräsentant
\begin{tightitemize}
  \item Einarbeitung neuer Community Mitglieder;
  \item Verwaltung und Updates von Abhängigkeiten in verschiedenen Versionen;
  \item Vertretung der Community auf Veranstaltungen und Konferenzen;
  \item Eskalation von Bugs und Featurerequests der Community an die Firma Puppet Inc.
\end{tightitemize}}

\tlcventry{2015}{0}{Puppet Contributor}{\href{https://puppet.com/}{Puppet Inc}}{Internet}{}{Development, bugfix and documentation
\begin{tightitemize}
 \item Review und Bearbeitung eingehender Patches;
 \item Umgang mit (Security-) Issues, Bugs und Feature Requests in über 90 Puppet Modulen;
 \item Migration öffentlicher Module von Puppet 3 nach 4, später 5;
\end{tightitemize}}

\tlcventry{2015}{0}{Developer and Maintainer}{\href{https://github.com/virtapi/virtapi}{VirtAPI-Stack}}{Internet}{}{Software Architekt und Entwickler
\begin{tightitemize}
  \item Entwurf und Dokumentation einer API zur Verwaltung einer Cloud-Platform;
  \item Verwaltung des installimages;
  \item Erstellung eines Live Linux Systems für Rescue Operationen (\href{https://github.com/virtapi/LARS}{LARS});
  \item Pflege einer API zur Verwaltung von PXE, DHCP, LDAP, installimage Daten (\href{https://github.com/virtapi/marmoset}{marmoset}).
\end{tightitemize}}

\section{Ausbildung}

\tllabelcventry{2014}{2018}{\kern-2em 2014--2018}{Staatlich geprüfter Informatiker}{\href{http://www.hhek.bonn.de/front_content.php?idcat=120}{Heinrich-Hertz-Europakolleg der Bundesstadt Bonn}}{}{}{}

\tllabelcventry{2009}{2012}{2009--2012}{Ausbildung zum Fachinformatiker Systemintegration}{}{Polizei NRW}{}{}{}

\tllabelcventry{2003}{2009}{2003--2009}{Realschulabschluss}{}{Realschule Heessen}{}{}{}

\section{Zertifizierung}

\tllabelcventry{2016}{2017}{\kern-3em 2016--2017}{Cisco Certified Network Associate}{\href{https://www.cisco.com/c/en/us/training-events/training-certifications/certifications/associate/ccna-routing-switching.html}{Cisco Inc}}{}{}{}

\tldatecventry{2017}{ITIL{\textregistered} v3 Foundation Examination}{\href{https://www.axelos.com/best-practice-solutions/itil}{AXELOS Limited}}{}{}{}

\tldatecventry{2013}{Puppet Fundamentals Administrator}{\href{https://puppet.com/support-services/training}{Puppet Inc}}{}{}{Puppet IT automation software}

\section{Veröffentlichungen}

\tldatecventry{2017}{Datawarehousing for Cloud Computing Metrics}{\href{https://bastelfreak.de/thesis-de.pdf}{https://bastelfreak.de/thesis-de.pdf}}{}{}{Analyse von Softwarekomponenten zur Messung von Cloud Metriken nach dem Datawarehouse Prinzip}{}

\tllabelcventry{2015}{2017}{\kern-2em 2015--2017}{Network Design Principles}{\href{https://github.com/bastelfreak/talks}{https://github.com/bastelfreak/talks}}{}{}{Taught multiple lessons about network design in datacenters and carriers, including routing, switching, network-security and IPv6}

\tldatecventry{2015}{VirtAPI Paper}{\href{https://bastelfreak.de/virtapi.pdf}{https://bastelfreak.de/virtapi.pdf}}{}{}{Studienarbeit über eine API für eine Cloud-Platform}{}

\tldatecventry{2012}{Entwicklung einer Testumgebung für Systemupdates}{\href{https://bastelfreak.de/doku.pdf}{https://bastelfreak.de/doku.pdf}}{}{}{Aufbau eines Qemu Hypervisors mit Archipel auf einer Gentoo Platform}{}

\section{Foreign Languages}

\cvlanguage{English}{Fliesend}{Tägliche Nutzung, Vorträge auf Englisch}

\end{document}
